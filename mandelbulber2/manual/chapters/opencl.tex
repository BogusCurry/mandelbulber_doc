

\section{OpenCL}\label{opencl}\index{OpenCL}

The use of OpenCL enables offloading the rendering of the fractal to the
GPU or to an accelerator card. This can highly reduce the render time.
OpenCL itself is an industry standard developed by the Khronos group and
is a well established framework. The two mayor GPU vendors (ATI / Nvidia)
among others implement the OpenCL specification in their drivers.

\subsection{Setup of OpenCL}\label{setup-opencl}
To render in Mandelbulber with OpenCL you will need to install a recent driver.
The newest GPU driver available can be obtained from the links in the next table.
Choose your operation system and misc settings.
Then download and install the driver image.

\begin{center}
	\begin{tabular}{ | l | r | }
		\hline
		AMD 	&
		\href{http://support.amd.com/en-us/download/}{http://support.amd.com/en-us/download/}
		\\ \hline
		Nvidia 	& 
		\href{http://www.nvidia.de/Download/index.aspx}{http://www.nvidia.de/Download/index.aspx}
		\\ \hline
		Intel	&
		\href{https://downloadcenter.intel.com}{https://downloadcenter.intel.com}
		\\ \hline
	\end{tabular}
\end{center}

You should also be able to use free drivers, if they support OpenCL.
Sadly the performance of those drivers is typically below the performance of the proprietary ones.

\subsubsection{Setup of OpenCL on Windows}\label{setup-opencl-windows}
With a recent driver, a capable GPU and the Mandelbulber OpenCL version you are already set.
Proceed with \ref{configure-opencl}.

\subsubsection{Setup of OpenCL on Linux}\label{setup-opencl-linux}
TODO

\subsubsection{Setup of OpenCL on MacOS}\label{setup-opencl-macos}
TODO

\subsubsection{Setup Trouble shooting}\label{setup-opencl-troubleshooting}
TODO

\subsection{Configuring OpenCL}\label{configure-opencl}
Open Mandelbulber and navigate to: Menu > File > Program Preferences > OpenCL (GPU).
You will find the configuration page in figure \ref{opencl_tab}.

\simpleImageWithCaptionFullWidth{img/manual/media/opencl_tab.png}
{OpenCL Tab in preferences}
{opencl_tab}

\begin{itemize}
	\item First you need to enable OpenCL by enabling the checkbox
	\item Then you need to select the platform and device to identify the OpenCL hardware 
		element to render with.
	\item \textbf{Precision} - Switch between float and double precision
	\begin{itemize}
		\item \textbf{float} - gives high performance render, but is limited in precision, 
			sometimes this mode can result in artifacts and bad DE.
		\item \textbf{double}- gives lower performance render (at least by factor 2), but has the same accuracy as the non OpenCL version. This mode may prevent artifacts.
	\end{itemize}
	\item \textbf{Mode} - Switches the rendering engine between different levels of shader extent.	
	\item \textbf{Memory Limit} - TODO
\end{itemize}
